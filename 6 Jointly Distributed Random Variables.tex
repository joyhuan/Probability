\documentclass{article}
\usepackage[utf8]{inputenc}
\linespread{1.3}
\title{6 Jointly Distributed Random Variables}
\author{Revisit \emph{A First Course in Probability}}
\date{April 2020}

\usepackage{natbib}
\usepackage{graphicx}
\usepackage{amsmath}
\usepackage{geometry}
 \geometry{
 a4paper,
 total={170mm,257mm},
 left=20mm,
 top=20mm,
 }
\begin{document}

\maketitle

\section*{Menu}
\begin{enumerate}
\item  Joint Distribution Functions 
\item  Independent RV
\item  Sums of Independent RVs
\item  Conditional Distributions: Discrete Case 
\item  Conditional Distributions: Continuous Case 
\item  Order Statistics
\item  Joint Probability Distribution of Functions of RVs
\item  Exchangeable RVs
\end{enumerate}


\section*{6.1 Joint Distribution Functions}
$F(a,b) = P(X \leq a, Y \leq b) -\inf < a, b < \inf$ 
\begin{align*}
F_X(a)  &= P(X \leq a) \\ 
        &= P(X \leq a, Y < \inf) \\ 
        &= P(\lim_{b \rightarrow \inf}(X \leq a, Y \leq b)) \\ 
        &= lim_{b \rightarrow \inf} P(X \leq a, Y \leq b) \\ 
        &= lim_{b \rightarrow \inf} F(a,b) \\ 
        &= F(a, \inf)
\end{align*}
$ F_Y(b) = F(\inf, b)$ \underline{\textbf{Marginal distributions of X and Y}} \\ 
All joint prob statements about X and Y can, in theory, be answered in terms of their joint distribution fcn. \\
For instance, joint prob that X is greater than a and Y is greater than b. 
\begin{align*}
P(X>a, Y>b) &= 1- P((X > a, Y > b)^C) \\ 
            &= 1- P((X > a)^C \cup (Y > b)^C) \\
            &= 1 - P((X \leq a) \cup (Y \leq b))  (1.1) \\
            &= 1 - [P(X \leq a) + P(Y \leq b) - P(X \leq a, Y \leq b)]\\
            &= 1 - F_X(a) - F_Y(b) + F(a,b) 
\end{align*}
$P(a_1 < X \leq a_2, b_1 < Y \leq b_2) = F(a_2, b_2) + F(a_1, b_1) - F(a_1, b_2) - F(a_2, b_1)$ (1.2)\\ 
[Makes much more sense when revisit it after two years haha] \\ 
In the case when X and Y are both discrete rvs, it is convenient to define the \underline{\textbf{joint prob mass fcn of X and Y}} by \\
$p(x,y) = P(X=x, Y=y)$ \\
The \underline{\textbf{prob mass fcn of X}} can be obtained from p(x,y) by \\
$ p_X(x) = P(X=x) = \sum_{y:p(x,y) > 0}p(x,y)$ \\
Similarly, \\
$ p_Y(y) = \sum_{x:p(x,y) > 0}p(x,y)$\\
Because the individual prob mass fcns of X and Y thus appear in the margin of such a table, they are often referred to as the \underline{\textbf{marginal prob mass fcns}} of X and Y, respectively. \\
We say that X and Y are \underline{\textbf{jointly continuous}} if there exists a fcn f(x,y), defined for all real x and y, having the property that, for every set C of pairs of real #s (that is, C is a set in the two-dimensional place),\\
$P((X,Y)\in C) = \int_{(x,y)\in C} f(x,y)dx dy $ (1.3)\\ 
The function f(x,y) is called the joint probability density function of X and Y. If A and B are any sets of real numbers, then, by defining $C=[(x,y):x \in A, y \in B]$, we see from Equation (1.3) that \\
$P(X \in A, Y \in B) = \int_{B}\int_{A}f(a,y)dx dy$ (1.4) \\ 
Because \\
$F(a,b) = P(X \in (-\inf, a), Y \in (-\inf, b)) = \int_{-\inf}^{b}\int_{-\inf}^{a}f(x,y)dx dy$ \\ 
it follows, upon differentiation, that \\
$f(a,b) = \frac{\partial^2}{\partial a \partial b}F(a, b)$\\
wherever the partial derivatives are defined. Another interpretation of the joint density function, obtained from Equation (1.4), is \\
[Tbh, from my experience, discrete is much more useful in real life] \\
$P(a < X < a + da, b < Y < b + db) = \int_{b}^{b + db} \inf_{a}^{a + da} f(x,y) dx dy$\\
when da and db are small and f(x,y) is continuous at a,b. Hence f(a,b) is the measure of how likely it is that the random vector (X,Y) will be near (a,b). \\
If X and Y are continuous, they are individually continuous, and their prob density fcns can be obtained as follows: 
\begin{align*}
P(X \in A)  &= P(X \in A, Y \in (-\inf, \inf)) \\ 
            &= \int_{A}\int_{-\inf}^{\inf}f(x,y)dy dx \\ 
            &= \int_{A}f_X(x)dx \\ 
where \\
f_X(x)      &= \int_{-\inf}^{\inf}f(x,y)dy
\end{align*}
is thus the prob density fcn of X. Similarly, the prob density fcn of Y is given by \\
$f_Y(y) = \int_{-\inf}^{\inf}f(x,y)dx$\\

% \begin{figure}[h!]
% \centering
% \includegraphics[scale=1.7]{universe}
% \caption{The Universe}
% \label{fig:universe}
% \end{figure}

% \section{Conclusion}
% ``I always thought something was fundamentally wrong with the universe'' \citep{adams1995hitchhiker}

\bibliographystyle{plain}
\bibliography{references}
\end{document}

