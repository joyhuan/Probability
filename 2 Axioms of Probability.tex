\documentclass{article}
\usepackage[utf8]{inputenc}
\linespread{1.3}
\title{2 Axioms of Probability}
\author{Revisit \emph{A First Course in Probability}}
\date{April 2020}

\usepackage{natbib}
\usepackage{graphicx}
\usepackage{amsmath}
\usepackage{geometry}
\usepackage{outlines}
 \geometry{
 a4paper,
 total={170mm,257mm},
 left=20mm,
 top=20mm,
 }
\begin{document}

\maketitle

\section*{Menu}
\begin{outline}[enumerate]
    \1  Introduction 
    \1  Sample Space and Events 
    \1  Axioms of Probability 
    \1  Some Simple Propositions  
    \1  Sample Spaces Having Equally Likely Outcomes 
    \1  Probability as a Continuous Set Function 
    \1  Probability as a Measure of Belief
\end{outline}


\section*{4.3 Axioms of Probability}
$P(E) = \lim_{n \rightarrow \inf} \frac{n(E)}{n}$\\
The three axioms of probability \\
Axiom 1 $0 \leq P(E) \leq 1$\\
Axiom 2 P(S) = 1 \\
Axiom 3 \\
For any sequence of mutually exclusive events $E_1, E_2, ...$ (that is, events for which $E_i E_j=\empty$ when $i \neq j$), \\
$P(\bigcup_{i=1}^{\inf}E_i) = \sum_{i=1}^{\inf}P(E_i)$\\
We refer to P(E) as the probability of the event E. 



% \begin{figure}[h!]
% \centering
% \includegraphics[scale=1.7]{universe}
% \caption{The Universe}
% \label{fig:universe}
% \end{figure}

% \section{Conclusion}
% ``I always thought something was fundamentally wrong with the universe'' \citep{adams1995hitchhiker}

\bibliographystyle{plain}
\bibliography{references}
\end{document}

