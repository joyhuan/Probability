\documentclass{article}
\usepackage[utf8]{inputenc}
\linespread{1.3}
\title{3 Conditional Probabilities and Independence}
\author{Revisit \emph{A First Course in Probability}}
\date{Feb 2021}

\usepackage{natbib}
\usepackage{graphicx}
\usepackage{amsmath}
\usepackage{geometry}
\usepackage{outlines}
\usepackage[T1]{fontenc}
 \geometry{
 a4paper,
 total={170mm,257mm},
 left=20mm,
 top=20mm,
 }
\begin{document}

\maketitle

\section*{Menu}
\begin{outline}[enumerate]
    \1  Introduction
    \1  Conditional Probabilities
    \1  Baye's Formula
    \1  Independent Events
    \1  $P(\dot | F)$ Is a Prob
\end{outline}


\section*{3.1 Introduction}
\section*{3.2 Conditional Probabilities}
\section*{3.3 Baye's Formula}
\begin{align*}
P(E) &= \sum_{i=1}^n P(EF_i) \\
     &= \sum_{i=1}^n P(E|F_i)P(F_i) (3.4)
\end{align*}
This formula is so far one of the most useful Rule I've used. 
\bibliographystyle{plain}
\bibliography{references}
\end{document}
